\section{Introduction}
This document is the developer guide to the \emph{PolyGame} project. The goal of this project is to provide a easy to use interface to play a game to increase the knowledge about the topic of sustainability and how it is possible to make the current \emph{$CO_2$} emission situation better.


The game idea comes from the work of Robert H. Socolow and Stephen W. Pacala described in \cite{scientificAmerican} and \cite{science}.

This project is based on the game proposed by professors Renato Casagrandi and Giulia Fiorese in the ASP Spring School 2009, as an activity for the course \emph{Global Change and Sustainability}

In this game students first have to study a single environmental topic, to find out the effort needed to reduce the related \emph{$CO_2$} emissions by 1 MegaTon. Each student or each group of students will be assigned to study a different topic to have all the possible issues covered by the class.

When the study part is over, students have to present to the class what are the result they obtained, pointing out what are the point of strength and the weakness of the proposed solution. From  the case they studied and the presentation of their class-mates, students should have a basic knowledge about all the topics.

In the second phase of the game students are asked to apply what they have just learned to create a plan to reduce the emissions by at least 20 MegaTons, using a combination of the solution studied before.

At the end of this phase students will present their environmental plan to the class, giving a brief overview of it and explaining the reasons of the choices.

The proposed plan as well as its presentation will be used by a panel of voters to decide the winner of the game

With \emph{PolyGame} we provide a web application that makes it easier to play the game in all its phases, making all the needed material available to the students and helping the organizer to follow what is going on.

In the next section of this document the technologies used to implement the application will be discussed and some implementation details will be provided to help the reader understanding how it works.