\section{Rules of the game}
\label{rules}
The aim of this game is to gain a good understanding of the possible technologies that can reduce CO$_{2}$ emissions and to be able to come up with a good mix of those to act against the climate crisis.\\

Players can either take part in the game singularly or in a group. In both cases the game is divided in two phases:
	\begin{description}
		\item[Phase one] It is the phase of the game when every single user and every group is assigned to a certain \emph{wedge}. A \emph{wedge} is a certain solution which can reduce CO$_{2}$ emissions by 1 MegaTon. Every group and every single user has to investigate the given wedge in order to find out pros and cons. At the end of this phase the group or the single user has to outline all the pros and cons in a \emph{poster}, which should be described in front of the audience in a few minutes (the exact number of minutes is decided by the organizer). The purpose of the presentation is to make everybody aware of the possible solutions to reduce CO$_{2}$ emissions.
		\item[Phase two] After the first phase every user might be assigned to a new group (while there might still be single user). Groups of phase two do not work a specific wedge but they have to come up with the best possible \emph{plan} to reduce emissions using a mix of the wedges studied in the previous phase. The minimum reduction of CO$_{2}$ should be at least be 20 MegaTons so a minimum quantity of 20 wedges is required. Any wedge can be selected more than once and this allows a wide range of possible solutions (and also a wide margin for discussion).
	\end{description}
	
	 At the end of phase two plans are voted and the most voted becomes the \emph{winner} of Polygame.\\
All comments left by voters to the voters are displayed.